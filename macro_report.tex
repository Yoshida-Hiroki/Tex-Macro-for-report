%\documentclass[a4j,dvipdfmx]{jsarticle}
\bibliographystyle{junsrt}
\usepackage[dvipdfmx]{graphicx}
\usepackage{amsmath,amssymb,physics,mathtools}
\usepackage{multicol}
\usepackage[dvipdfmx]{color}
\usepackage{url}
\usepackage[dvipdfmx]{hyperref}
\usepackage{pxjahyper}
\usepackage{wrapfig}
\usepackage{tikz}
\usepackage[top=10truemm,bottom=15truemm,left=25truemm,right=18truemm]{geometry}
\usetikzlibrary{decorations.markings}
\date{}

% サブセクションを 問1,問2 にする設定
\renewcommand{\thesubsection}{問題\arabic{section}.\arabic{subsection}}

% サブサブセクションを (1),(2)にする設定
%\renewcommand{\thesubsubsection}{(\arabic{subsubsubsection})}
% (i),(ii)なら \arabic を \roman に変える。    (a),(b)なら \alph

% 大問2の3番目の計算式のラベルを (2.3) にする設定
% 計算式の参照には \eqref{eq:hoge} を使う
\makeatletter
  \renewcommand{\theequation}{\arabic{section}.\arabic{subsection}.\arabic{equation}}
  \@addtoreset{equation}{subsection}
\makeatother

\makeatletter
\def\@cite#1#2{$^{\mbox{\tiny[{#1\if@tempswa , #2\fi}]}}$}
\makeatother
\makeatletter
\renewcommand{\@cite}[1]{\textsuperscript{#1)}}
\renewcommand{\@biblabel}[1]{#1)}
\makeatother
\newcommand{\Cross}{$\mathbin{\tikz [x=1.4ex,y=1.4ex,line width=.2ex, red] \draw (0,0) -- (1,1) (0,1) -- (1,0);}$}%
\def\st#1\en{\begin{align*}#1\end{align*}}
